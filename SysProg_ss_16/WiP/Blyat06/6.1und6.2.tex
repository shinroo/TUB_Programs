
\documentclass{article}
\makeatletter
\usepackage[table,xcdraw]{xcolor}
\usepackage{graphicx}
\usepackage{placeins}
\usepackage{caption} % for better vertical separation
\usepackage{hyperref}
\usepackage{hypcap} % fix the links
\usepackage[margin=0.5in]{geometry}
\usepackage[ansinew]{inputenc}
\usepackage[T1]{fontenc}
\usepackage{amsfonts}
\usepackage[normalem]{ulem}
\useunder{\uline}{\ul}{}
\begin{document}

\LARGE{Systemprogrammierung, SS16}	

\large{\"Ubungblatt 6, Theorie} \\


\textbf{Gruppe}:
\begin{itemize}
	\item Robert Focke
	\item Hristo Filaretov
	\item Mikolaj Walukiewicz
\end{itemize}


\textbf{Quellen}:
\begin{itemize}
	\item Kao, Odej. "Systemprogrammierung". 2016. Vorlesungsfolien "6. Speicherverwaltung".
\end{itemize}


\section{Aufgabe}
Relevante Tabelle: \ref{Optimal}, \ref{FIFO}, \ref{LFU}, \ref{LRU}

\begin{table}[h]
	\caption{Optimale L\"osung - nur 8 Seitenfehler}
	\centering
	\label{Optimal}
	\begin{tabular}{|l|lllllllllllllll|}
		\hline
		Seite    & 1                & 2                & 3                & 4                & 3          & 5                & 1          & 6                & 2          & 6          & 5          & 4                & 3                & 2          & 1          \\ \hline
		Kachel 1 & {\ul \textbf{1}} & 1                & 1                & 1                & 1          & 1                & \textbf{1} & {\ul \textbf{6}} & 6          & \textbf{6} & 6          & 6                & {\ul \textbf{3}}                & 3          & 3 \\ \cline{1-1}
		Kachel 2 & -                & {\ul \textbf{2}} & 2                & 2                & 2          & 2                & 2          & 2                & \textbf{2} & 2          & 2          & 2                & 2                & \textbf{2} & 2          \\ \cline{1-1}
		Kachel 3 & -                & -                & {\ul \textbf{3}} & 3                & \textbf{3} & {\ul \textbf{5}} & 5          & 5                & 5          & 5          & \textbf{5} & 5	 & 5 & 5          & {\ul \textbf{1} }          \\ \cline{1-1}
		Kachel 4 & -                & -                & -                & {\ul \textbf{4}} & 4          & 4                & 4          & 4 & 4          & 4 & 4          & \textbf{4}                & 4 & 4          & 4          \\ \hline
	\end{tabular}
\end{table}

\begin{table}[h]
	\caption{FIFO mit dem  n\"achsten auszulagernden Kachel - insgesamt 11 Seitenzugriffsfehler.}
	\centering
	\label{FIFO}
	\begin{tabular}{|l|lllllllllllllll|}
		\hline
		Seite    & 1                & 2                & 3                & 4                & 3          & 5                & 1                & 6                & 2                & 6          & 5          & 4                & 3                & 2          & 1                \\ \hline
		Kachel 1 & {\ul \textbf{1}} & 1                & 1                & 1                & 1          & {\ul \textbf{5}} & 5                & 5                & 5                & 5          & \textbf{5} & {\ul \textbf{4}} & 4                & 4          & 4                \\ \cline{1-1}
		Kachel 2 & -                & {\ul \textbf{2}} & 2                & 2                & 2          & 2                & {\ul \textbf{1}} & 1                & 1                & 1          & 1          & 1                & {\ul \textbf{3}} & 3          & 3                \\ \cline{1-1}
		Kachel 3 & -                & -                & {\ul \textbf{3}} & 3                & \textbf{3} & 3                & 3                & {\ul \textbf{6}} & 6                & \textbf{6} & 6          & 6                & 6                & 6          & {\ul \textbf{1}} \\ \cline{1-1}
		Kachel 4 & -                & -                & -                & {\ul \textbf{4}} & 4          & 4                & 4                & 4                & {\ul \textbf{2}} & 2          & 2          & 2                & 2                & \textbf{2} & 2                \\ \hline
		N. a. K. & 1                & 1                & 1                & 1                & 1          & 2                & 3                & 4                & 5                & 5          & 5          & 1                & 6                & 6          & 2                \\ \hline
	\end{tabular}
\end{table}


% Please add the following required packages to your document preamble:
% \usepackage[normalem]{ulem}
% \useunder{\uline}{\ul}{}
\begin{table}[h]
	\caption{LFU - insgesamt 12 Seitenzugriffsfehler}
	\label{LFU}
	\centering
	\begin{tabular}{|l|lllllllllllllll|}
		\hline
		\multicolumn{1}{|l|}{Seite}    & 1                & 2                & 3                & 4                & 3          & 5                & 1                & 6                & 2                & 6          & 5                & 4                & 3          & 2                & 1                \\ \hline
		\multicolumn{1}{|l|}{Kachel 1} & {\ul \textbf{1}} & 1                & 1                & 1                & 1          & {\ul \textbf{5}} & 5                & 5                & {\ul \textbf{2}} & 2          & 2                & {\ul \textbf{4}} & 4          & 4                & {\ul \textbf{1}} \\ \cline{1-1}
		\multicolumn{1}{|l|}{Kachel 2} & -                & {\ul \textbf{2}} & 2                & 2                & 2          & 2                & {\ul \textbf{1}} & 1                & 1                & 1          & {\ul \textbf{5}} & 5                & 5          & {\ul \textbf{2}} & 2                \\ \cline{1-1}
		\multicolumn{1}{|l|}{Kachel 3} & -                & -                & {\ul \textbf{3}} & 3                & \textbf{3} & 3                & 3                & 3                & 3                & 3          & 3                & 3                & \textbf{3} & 3                & 3                \\ \cline{1-1}
		\multicolumn{1}{|l|}{Kachel 4} & -                & -                & -                & {\ul \textbf{4}} & 4          & 4                & 4                & {\ul \textbf{6}} & 6                & \textbf{6} & 6                & 6                & 6          & 6                & 6                \\ \hline
		LFU Queue                      & $1_1$              & $2_1$             & $3_1$             & $4_1$              & $3_2$       & $3_2$              & $3_2$              & $3_2$              & $3_2$              & $6_2$        & $6_2 $             & $6_2$              & $3_3$        & $3_3$              & $3_3$      \\
		&                  & $1_1$               & $2_1$             & $3_1$              & $4_1$        & $5_1$              & $1_1$               & $6_1$              & $2_1$             & $3_2$        & $3_2$              & $3_2$              & $6_2$         & $6_2$               & $6_2$               \\
		&                  &                  & $1_1$               & $2_1$              & $2_1$        & $4_1$              & $5_1$              & $1_1$               & $6_1$              & $2_1$        & $5_1$              & $4_1$              & $4_1$        & $2_1$             & $1_1$               \\
		&                  &                  &                  & $1_1$               & $1_1$         & $2_1$              & $4_1$              & $5_1$              & $1_1$              & $1_1$         & $2_1$             & $5_1$             & $5_1$       & $4_1$              & $2_1$             \\ \hline
	\end{tabular}
\end{table}

\begin{table}[h]
	\caption{LRU - insgesamt 12 Seitenzugriffsfehler.}
	\centering
	\label{LRU}
	\begin{tabular}{|l|lllllllllllllll|}
		\hline
		\multicolumn{1}{|l|}{Seite}    & 1                & 2                & 3                & 4                & 3          & 5                & 1                & 6                & 2                & 6          & 5          & 4                & 3                & 2                & 1                \\ \hline
		\multicolumn{1}{|l|}{Kachel 1} & {\ul \textbf{1}} & 1                & 1                & 1                & 1          & {\ul \textbf{5}} & 5                & 5                & 5                & 5          & \textbf{5} & 5                & 5                & 5                & {\ul \textbf{1}} \\ \cline{1-1}
		\multicolumn{1}{|l|}{Kachel 2} & -                & {\ul \textbf{2}} & 2                & 2                & 2          & 2                & {\ul \textbf{1}} & 1                & 1                & 1          & 1          & {\ul \textbf{4}} & 4                & 4                & 4                \\ \cline{1-1}
		\multicolumn{1}{|l|}{Kachel 3} & -                & -                & {\ul \textbf{3}} & 3                & \textbf{3} & 3                & 3                & 3                & {\ul \textbf{2}} & 2          & 2          & 2                & {\ul \textbf{3}} & 3                & 3                \\ \cline{1-1}
		\multicolumn{1}{|l|}{Kachel 4} & -                & -                & -                & {\ul \textbf{4}} & 4          & 4                & 4                & {\ul \textbf{6}} & 6                & \textbf{6} & 6          & 6                & 6                & {\ul \textbf{2}} & 2                \\ \hline
		LRU Queue                      & 1                & 2                & 3                & 4                & 3          & 5                & 1                & 6                & 2                & 6          & 5          & 4                & 3                & 2                & 1                \\
		&                  & 1                & 2                & 3                & 4          & 3                & 5                & 1                & 6                & 2          & 6          & 5                & 4                & 3                & 2                \\
		&                  &                  & 1                & 2                & 2          & 4                & 3                & 5                & 1                & 1          & 2          & 6                & 5                & 4                & 3                \\
		&                  &                  &                  & 1                & 1          & 2                & 4                & 3                & 5                & 5          & 1          & 2                & 6                & 5                & 4               \\ \hline
	\end{tabular}
\end{table}
\FloatBarrier
\section{Aufgabe}

Relevante Tabelle: \ref{ffit}, \ref{nfit}, \ref{bfit}, \ref{wfit}

\begin{table}[h]
	\caption{First fit}
	\centering
	\label{ffit}
	\begin{tabular}{|l|
			>{\columncolor[HTML]{343434}}l |l|l|
			>{\columncolor[HTML]{343434}}l |l|l|
			>{\columncolor[HTML]{343434}}l |l|l|
			>{\columncolor[HTML]{343434}}l |c|l|
			>{\columncolor[HTML]{343434}}l |}
		\hline
		\multicolumn{1}{|c|}{command} &  & \multicolumn{2}{c|}{5} &  & \multicolumn{2}{c|}{8} &  & \multicolumn{2}{c|}{14} &  & \multicolumn{2}{c|}{6} &  \\ \hline
		A1=6 &  & \multicolumn{2}{c|}{5} &  & A1 & \cellcolor[HTML]{FFFFFF}{\color[HTML]{000000} 2} &  & \multicolumn{2}{c|}{14} &  & \multicolumn{2}{c|}{6} &  \\ \hline
		A2=3 &  & A2 & 2 &  & A1 & \cellcolor[HTML]{FFFFFF}{\color[HTML]{333333} 2} &  & \multicolumn{2}{c|}{14} &  & \multicolumn{2}{c|}{6} &  \\ \hline
		A3=10 &  & A2 & 2 &  & A1 & \cellcolor[HTML]{FFFFFF}{\color[HTML]{333333} 2} &  & A3 & 4 &  & \multicolumn{2}{c|}{6} &  \\ \hline
		free(A1) &  & A2 & 2 &  & \multicolumn{2}{c|}{8} &  & A3 & 4 &  & \multicolumn{2}{c|}{6} &  \\ \hline
		A4=7 & {\color[HTML]{000000} } & A2 & 2 & {\color[HTML]{000000} } & A4 & \cellcolor[HTML]{FFFFFF}{\color[HTML]{333333} 1} &  & A3 & 4 &  & \multicolumn{2}{c|}{6} &  \\ \hline
		A5=5 &  & A2 & 2 &  & A4 & \cellcolor[HTML]{FFFFFF}{\color[HTML]{333333} 1} &  & A3 & 4 &  & \multicolumn{1}{l|}{A5} & 1 &  \\ \hline
		free(A3) &  & A2 & 2 &  & A4 & \cellcolor[HTML]{FFFFFF}{\color[HTML]{333333} 1} &  & \multicolumn{2}{c|}{14} &  & \multicolumn{1}{l|}{A5} & 1 &  \\ \hline
		A6=12 & \cellcolor[HTML]{333333} & A2 & 2 & \cellcolor[HTML]{333333} & A4 & 1 & \cellcolor[HTML]{333333} & A6 & 2 & \cellcolor[HTML]{333333} & \multicolumn{1}{l|}{A5} & 1 & \cellcolor[HTML]{333333} \\ \hline
	\end{tabular}
\end{table}
% Please add the following required packages to your document preamble:
% \usepackage[table,xcdraw]{xcolor}
% If you use beamer only pass "xcolor=table" option, i.e. \documentclass[xcolor=table]{beamer}
\begin{table}[h]
	\centering
	\caption{Next fit}
	\label{nfit}
	\begin{tabular}{|l|
			>{\columncolor[HTML]{343434}}l |c|l|
			>{\columncolor[HTML]{343434}}l |l|l|
			>{\columncolor[HTML]{343434}}l |l|l|l|
			>{\columncolor[HTML]{343434}}l |c|l|
			>{\columncolor[HTML]{343434}}l |}
		\hline
		\multicolumn{1}{|c|}{command}                                         &                          & \multicolumn{2}{c|}{5} &                          & \multicolumn{2}{c|}{8}                                &                          & \multicolumn{3}{c|}{14}                           &                          & \multicolumn{2}{c|}{6}      &                          \\ \hline
		A1=6                                                            &                          & \multicolumn{2}{c|}{5} &                          & A1 & \cellcolor[HTML]{FFFFFF}{\color[HTML]{000000} 2} &                          & \multicolumn{3}{c|}{14}                           &                          & \multicolumn{2}{c|}{6}      &                          \\ \hline
		A2=3                                                            &                          & \multicolumn{2}{c|}{5} &                          & A1 & \cellcolor[HTML]{FFFFFF}{\color[HTML]{333333} 2} &                          & \multicolumn{1}{c|}{A2} & \multicolumn{2}{c|}{11} &                          & \multicolumn{2}{c|}{6}      &                          \\ \hline
		A3=10                                                           &                          & \multicolumn{2}{c|}{5} &                          & A1 & \cellcolor[HTML]{FFFFFF}{\color[HTML]{333333} 2} &                          & A2                      & A3          & 1         &                          & \multicolumn{2}{c|}{6}      &                          \\ \hline
		free(A1)                                                        &                          & \multicolumn{2}{c|}{5} &                          & \multicolumn{2}{c|}{8}                                &                          & A2                      & A3          & 1         &                          & \multicolumn{2}{c|}{6}      &                          \\ \hline
		A4=7                                                            & {\color[HTML]{000000} }  & \multicolumn{2}{c|}{5} & {\color[HTML]{000000} }  & A4 & \cellcolor[HTML]{FFFFFF}{\color[HTML]{333333} 1} &                          & A2                      & A3          & 1         &                          & \multicolumn{2}{c|}{6}      &                          \\ \hline
		A5=5                                                            &                          & \multicolumn{2}{c|}{5} &                          & A4 & \cellcolor[HTML]{FFFFFF}{\color[HTML]{333333} 1} &                          & A2                      & A3          & 1         &                          & \multicolumn{1}{l|}{A5} & 1 &                          \\ \hline
		free(A3)                                                        &                          & \multicolumn{2}{c|}{5} &                          & A4 & \cellcolor[HTML]{FFFFFF}{\color[HTML]{333333} 1} &                          & \multicolumn{1}{c|}{A2} & \multicolumn{2}{c|}{11} &                          & \multicolumn{1}{l|}{A5} & 1 &                          \\ \hline
		\begin{tabular}[c]{@{}l@{}}A6=12\\ (Geht nicht!!!)\end{tabular} & \cellcolor[HTML]{333333} & \multicolumn{2}{c|}{5} & \cellcolor[HTML]{333333} & A4 & 1                                                & \cellcolor[HTML]{333333} & A2                      & \multicolumn{2}{c|}{11} & \cellcolor[HTML]{333333} & \multicolumn{1}{l|}{A5} & 1 & \cellcolor[HTML]{333333} \\ \hline
	\end{tabular}
\end{table}
% Please add the following required packages to your document preamble:
% \usepackage[table,xcdraw]{xcolor}
% If you use beamer only pass "xcolor=table" option, i.e. \documentclass[xcolor=table]{beamer}
\begin{table}[h]
	\centering
	\caption{Best fit}
	\label{bfit}
	\begin{tabular}{|l|
			>{\columncolor[HTML]{343434}}l |c|l|l|
			>{\columncolor[HTML]{343434}}l |c|l|
			>{\columncolor[HTML]{343434}}l |l|c|
			>{\columncolor[HTML]{343434}}l |c|l|
			>{\columncolor[HTML]{343434}}l |}
		\hline
		\multicolumn{1}{|c|}{command} &                          & \multicolumn{3}{c|}{5}           &                          & \multicolumn{2}{c|}{8}                                                     &                          & \multicolumn{2}{c|}{14} &                          & \multicolumn{2}{c|}{6}      &                          \\ \hline
		A1=6                    &                          & \multicolumn{3}{c|}{5}           &                          & \multicolumn{2}{c|}{8}                                                     &                          & \multicolumn{2}{c|}{14} &                          & \multicolumn{2}{c|}{A1}     &                          \\ \hline
		A2=3                    &                          & \multicolumn{2}{c|}{A2} & 2      &                          & \multicolumn{2}{c|}{8}                                                     &                          & \multicolumn{2}{c|}{14} &                          & \multicolumn{2}{c|}{A1}     &                          \\ \hline
		A3=10                   &                          & \multicolumn{2}{c|}{A2} & 2      &                          & \multicolumn{2}{c|}{8}                                                     &                          & A3          & 4         &                          & \multicolumn{2}{c|}{A1}     &                          \\ \hline
		free(A1)                &                          & \multicolumn{2}{c|}{A2} & 2      &                          & \multicolumn{2}{c|}{8}                                                     &                          & A3          & 4         &                          & \multicolumn{2}{c|}{6}      &                          \\ \hline
		A4=7                    & {\color[HTML]{000000} }  & \multicolumn{2}{c|}{A2} & 2      & {\color[HTML]{000000} }  & \multicolumn{1}{l|}{A4} & \cellcolor[HTML]{FFFFFF}{\color[HTML]{333333} 1} &                          & A3          & 4         &                          & \multicolumn{2}{c|}{6}      &                          \\ \hline
		A5=5                    &                          & \multicolumn{2}{c|}{A2} & 2      &                          & \multicolumn{1}{l|}{A4} & \cellcolor[HTML]{FFFFFF}{\color[HTML]{333333} 1} &                          & A3          & 4         &                          & \multicolumn{1}{l|}{A5} & 1 &                          \\ \hline
		free(A3)                &                          & \multicolumn{2}{c|}{A2} & 2      &                          & \multicolumn{1}{l|}{A4} & \cellcolor[HTML]{FFFFFF}{\color[HTML]{333333} 1} &                          & \multicolumn{2}{c|}{14} &                          & \multicolumn{1}{l|}{A5} & 1 &                          \\ \hline
		A6=12                   & \cellcolor[HTML]{333333} & \multicolumn{2}{c|}{A2} & 2      & \cellcolor[HTML]{333333} & \multicolumn{1}{l|}{A4} & 1                                                & \cellcolor[HTML]{333333} & A6          & 2         & \cellcolor[HTML]{333333} & \multicolumn{1}{l|}{A5} & 1 & \cellcolor[HTML]{333333} \\ \hline
	\end{tabular}
\end{table}
% Please add the following required packages to your document preamble:
% \usepackage[table,xcdraw]{xcolor}
% If you use beamer only pass "xcolor=table" option, i.e. \documentclass[xcolor=table]{beamer}
\begin{table}[]
	\centering
	\caption{Worst fit}
	\label{wfit}
	\begin{tabular}{|l|
			>{\columncolor[HTML]{343434}}l |c|l|l|
			>{\columncolor[HTML]{343434}}l |c|l|
			>{\columncolor[HTML]{343434}}l |c|c|l|
			>{\columncolor[HTML]{343434}}l |c|l|
			>{\columncolor[HTML]{343434}}l |}
		\hline
		\multicolumn{1}{|c|}{command} &                          & \multicolumn{3}{c|}{5} &                          & \multicolumn{2}{c|}{8}                                                     &                          & \multicolumn{3}{c|}{14}                          &                          & \multicolumn{2}{c|}{6} &                          \\ \hline
		A1=6                          &                          & \multicolumn{3}{c|}{5} &                          & \multicolumn{2}{c|}{8}                                                     &                          & A1                      & \multicolumn{2}{c|}{8} &                          & \multicolumn{2}{c|}{6} &                          \\ \hline
		A2=3                          &                          & \multicolumn{3}{c|}{5} &                          & A2                      & \cellcolor[HTML]{FFFFFF}{\color[HTML]{333333} 5} &                          & A1                      & \multicolumn{2}{c|}{8} &                          & \multicolumn{2}{c|}{6} &                          \\ \hline
		A3=10(Geht nicht!!!)          &                          & \multicolumn{3}{c|}{5} &                          & A2                      & \cellcolor[HTML]{FFFFFF}{\color[HTML]{333333} 5} &                          & \multicolumn{1}{l|}{A1} & \multicolumn{2}{c|}{8} &                          & \multicolumn{2}{c|}{6} &                          \\ \hline
		free(A1)                      &                          & \multicolumn{3}{c|}{5} &                          & A2                      & \cellcolor[HTML]{FFFFFF}{\color[HTML]{333333} 5} &                          & \multicolumn{3}{c|}{14}                          &                          & \multicolumn{2}{c|}{6} &                          \\ \hline
		A4=7                          & {\color[HTML]{000000} }  & \multicolumn{3}{c|}{5} & {\color[HTML]{000000} }  & \multicolumn{1}{l|}{A2} & \cellcolor[HTML]{FFFFFF}{\color[HTML]{333333} 5} &                          & \multicolumn{2}{l|}{A4}              & 7         &                          & \multicolumn{2}{c|}{6} &                          \\ \hline
		A5=5                          &                          & \multicolumn{3}{c|}{5} &                          & \multicolumn{1}{l|}{A2} & \cellcolor[HTML]{FFFFFF}{\color[HTML]{333333} 5} &                          & \multicolumn{1}{l|}{A4} & A5         & 2         &                          & \multicolumn{2}{c|}{6} &                          \\ \hline
		free(A3)(entf\"allt)            &                          & \multicolumn{3}{c|}{5} &                          & \multicolumn{1}{l|}{A2} & \cellcolor[HTML]{FFFFFF}{\color[HTML]{333333} 5} &                          & A4                      & A5         & 2         &                          & \multicolumn{2}{c|}{6} &                          \\ \hline
		A6=12(Geht nicht!!!)            & \cellcolor[HTML]{333333} & \multicolumn{3}{c|}{5} & \cellcolor[HTML]{333333} & \multicolumn{1}{l|}{A2} & 5                                                & \cellcolor[HTML]{333333} & \multicolumn{1}{l|}{A4} & A5         & 2         & \cellcolor[HTML]{333333} & \multicolumn{2}{c|}{6} & \cellcolor[HTML]{333333} \\ \hline
	\end{tabular}
\end{table}

\section{Aufgabe}

Relevante Tabelle: \ref{FAT3}

\begin{table}[]
	\centering
	\caption{FAT Tabelle}
	\label{FAT3}
	\begin{tabular}{|c|c|}
		\hline
		Block & N\"achster Block \\ \hline
		0 & 5 \\ \hline
		1 & 2 \\ \hline
		2 & NULL \\ \hline
		3 & NULL\\ \hline
	Start C: 4 & 6 \\ \hline
		5 & 19 \\ \hline
		6 & 7 \\ \hline
		7 & 20 \\ \hline
		8 & NULL \\ \hline
	Start A: 9 & 17 \\ \hline
		10 & 0 \\ \hline
		11 & NULL \\ \hline
		12 & NULL \\ \hline
	Start B: 13 & 10 \\ \hline
		14 & NULL \\ \hline
		15 & 12 \\ \hline
	Start D: 16 & 15 \\ \hline
		17 & 1 \\ \hline
		18 & NULL \\ \hline
		19 & 8 \\ \hline
		20 & NULL \\ \hline
	\end{tabular}
\end{table}

\section{Aufgabe}

Die in Aufgabe 6.4 gegebene FAT Tabelle enth�lt folgende Dateien:

$ A: 4 \rightarrow 6 \rightarrow 2 \rightarrow NULL $

$ B: 3 \rightarrow 0 \rightarrow 5 \rightarrow 8 \rightarrow 7 \rightarrow NULL $

$ C: 1 \rightarrow 9 \rightarrow 5 \rightarrow 8 \rightarrow 7 \rightarrow NULL $

Das Problem ist das die Dateien B und C zum Teil \"Uberlappen.
Wenn die Dateien nicht ge�ndert oder verschoben werden, ist das in Ordnung,
aber sobald eine der zwei Dateien ge�ndert oder verschoben wird kann es dazu f\"uhren,
dass die andere Datei unlesbar oder korrupt wird.

\end{document}
